\documentclass[times,]{simauth}
\usepackage{moreverb}
\usepackage[colorlinks,bookmarksopen,bookmarksnumbered,citecolor=red,urlcolor=red]{hyperref}
\usepackage[numbers, sort&compress]{natbib}
\def\volumeyear{2017}

\begin{document}

\runninghead{Uthor \emph{et al}.}

\title{A demonstration of the \LaTeX class file for Statistics in Medicine with
Rmarkdown}
\author{A. Uthor\affilnum{a,b}, O. Tro\affilnum{b} and O. Vriga\affilnum{c}}
\address{\affilnum{a} Department of Incredible Research, University A\\\affilnum{b} Department of Applied Things, University B\\\affilnum{c} Very Important Stuff Committee, Institute C\\}
\corraddr{\href{mailto:some@academic.email.org}{\nolinkurl{some@academic.email.org}}}

\begin{abstract}
This paper describes the use of the 
aTeX \texttt{simauth.cls} class
file for setting papers for Statistics in Medicine using Rmarkdown.
\end{abstract}

\keywords{Class file; \LaTeX; Statist. Med.; Rmarkdown;}

\maketitle

\section{Introduction}\label{introduction}

Many authors submitting to research journals use \LaTeX to prepare their
papers. This paper describes the \texttt{simauth.cls} class file which
can be used to convert articles produced with other \LaTeX class files
into the correct form for publication in \emph{Statistics in Medicine}.

The \texttt{simauth.cls} class file preserves much of the standard
\LaTeX interface so that any document which was produced using the
standard \LaTeX \texttt{article} style can easily be converted to work
with the \texttt{simauth} style. However, the width of text and typesize
will vary from that of \texttt{article.cls}; therefore, \emph{line
breaks will change} and it is likely that displayed mathematics and
tabular material will need re-setting.

In the following sections we describe how to lay out your code to use
\texttt{simauth.cls} to reproduce the typographical look of
\emph{Statistics in Medicine}. However, this paper is not a guide to
using \LaTeX and we would refer you to any of the many books available
(see, for example, \citep{R1, R2, R3}).

\section{The Three Golden Rules}\label{the-three-golden-rules}

Before we proceed, we would like to stress \emph{three golden rules}
that need to be followed to enable the most efficient use of your code
at the typesetting stage:

\begin{enumerate}
\def\labelenumi{\arabic{enumi}.}
\item
  keep your own macros to an absolute minimum;
\item
  as \texttt{TeX} is designed to make sensible spacing decisions by
  itself, do \emph{not} use explicit horizontal or vertical spacing
  commands, except in a few accepted (mostly mathematical) situations,
  such as \texttt{\textbackslash{},} before a
  differential\textasciitilde{}d, or \texttt{\textbackslash{}quad} to
  separate an equation from its qualifier;
\item
  follow the \emph{Statistics in Medicine} reference style.
\end{enumerate}

\section{Getting Started}\label{getting-started}

The \texttt{simauth} class file should run on any standard
\LaTeX installation. If any of the fonts, class files or packages it
requires are missing from your installation, they can be found on the
\emph{TeX Collection} DVDs or from CTAN.

Details on Rmarkdown can be found online
\href{here}{http://rmarkdown.rstudio.com/}.

\emph{Statistics in Medicine} is published using Times fonts and this is
achieved by using the \texttt{times} option as
\texttt{\textbackslash{}documentclass{[}times{]}\{simauth\}}. Times
fonts are also used for mathematics. This is achieved by adding the
\LaTeX package \texttt{mathtime}. Being \texttt{mathtime} not available
on \TeX Live installations, the default template does not include it. If
you need/want to re-enable it, add the option \texttt{keep\_tex:\ TRUE}
to the YAML header as follows and edit the resulting \texttt{.tex} file
manually.

\begin{verbatim}
output:
  rticles::sim_article:
    keep_tex: TRUE
\end{verbatim}

If for any reason you have a problem using Times you can easily resort
to Computer Modern fonts by removing the \texttt{times} option.

\section{The Article Header
Information}\label{the-article-header-information}

Configure the YAML header including the following elements:

\begin{itemize}
\item
  \texttt{title}: Title
\item
  \texttt{author}: Author(s) information, as a string, see below
\item
  \texttt{address}: List containing \texttt{address} and \texttt{num}
  for defining \texttt{author} affiliations
\item
  \texttt{corraddr}: Corresponding author address
\item
  \texttt{authabbr}: Short author list for header
\item
  \texttt{date}: Date of submission
\item
  \texttt{year}: Year of submission
\item
  \texttt{abstract}: Limited to 250 words
\item
  \texttt{keywords}: Up to 6 keywords
\item
  \texttt{bibliography}: Bibtex \texttt{.bib} file
\end{itemize}

\subsection{Author information}\label{author-information}

In order to obtain better results, author(s) information should be
provided as a string with \LaTeX elements:

\begin{verbatim}
author: "A.U. Thor\\affilnum{a,b}, O. Tro\\affilnum{b} and O. Vriga\\affilnum{c}"
\end{verbatim}

\subsection{Remarks}\label{remarks}

\begin{enumerate}
\def\labelenumi{\arabic{enumi}.}
\item
  In \texttt{authabbr} use \emph{et al.} if there are three or more
  authors.
\item
  Note the use of \texttt{affilnum} and \texttt{num} to link names and
  addresses. The author for correspondence can be marked via a custom
  \texttt{address} entry and \texttt{corraddr} is used to give that
  author's address, which will be printed as a footnote, prefaced by
  \emph{Correspondence to:}.
\item
  For submitting a double-spaced manuscript, add \texttt{doublespace} as
  an option to a \texttt{classoption} line in the YAML header:
  \texttt{classoption:\ doublespace}.
\item
  Use \texttt{\textbackslash{}cgs} for giving details of financial
  sponsors. These details will be printed as a footnote, with
  \emph{Contract/grant sponsor:} inserted in the appropriate places.
  This has to be implemented using \LaTeX, and not Rmarkdown.
\item
  The abstract should be capable of standing by itself, in the absence
  of the body of the article and of the bibliography. Therefore, it must
  not contain any reference citations.
\item
  Keywords are separated by semicolons.
\end{enumerate}

\section{The Body of the Article}\label{the-body-of-the-article}

\subsection{Mathematics}\label{mathematics}

\texttt{simauth.cls} makes the full functionality of \texttt{AmS/TeX}
available. We encourage the use of the \texttt{align}, \texttt{gather}
and \texttt{multline} environments for displayed mathematics.

Use mathematics in Rmarkdown as usual.

\subsection{Figures and Tables}\label{figures-and-tables}

\texttt{simauth.cls} uses the \texttt{graphicx} package for handling
figures.

Figures are supported from R code too:

\begin{verbatim}
x = rnorm(10)
y = rnorm(10)
plot(x, y)
\end{verbatim}

\begin{figure}
\centering
\includegraphics{StatsMedExample_files/figure-latex/plot_ref-1.pdf}
\caption{Fancy Caption\label{fig:plot}}
\end{figure}

\ldots{}and can be referenced: \ref{fig:plot}. It is a quirky hack at
the moment, see \href{here}{https://github.com/yihui/knitr/issues/323}.

Analogously, use Rmarkdown to produce tables as usual:

\begin{verbatim}
library(xtable)
xtable(head(cars), caption = "A table", label = "tab:table")
\end{verbatim}

\% latex table generated in R 3.4.3 by xtable 1.8-2 package \% Tue Mar
06 09:34:36 2018

\begin{table}[ht]
\centering
\begin{tabular}{rrr}
  \hline
 & speed & dist \\ 
  \hline
1 & 4.00 & 2.00 \\ 
  2 & 4.00 & 10.00 \\ 
  3 & 7.00 & 4.00 \\ 
  4 & 7.00 & 22.00 \\ 
  5 & 8.00 & 16.00 \\ 
  6 & 9.00 & 10.00 \\ 
   \hline
\end{tabular}
\caption{A table} 
\label{tab:table}
\end{table}

Referenced via \ref{tab:table}.

\subsection{Cross-referencing}\label{cross-referencing}

The use of the Rmarkdown equivalent of the \LaTeX cross-reference system
for figures, tables, equations, etc., is encouraged (using
\texttt{{[}@\textless{}name\textgreater{}{]}}, equivalent of
\texttt{\textbackslash{}ref\{\textless{}name\textgreater{}\}} and
\texttt{\textbackslash{}label\{\textless{}name\textgreater{}\}}). That
works well for citations in Rmarkdown, not so well for figures and
tables. In that case, it is possible to revert to standard
\LaTeX syntax.

Example: \citep{pepe2003statistical, zou2005regularization}.

\subsection{Acknowledgements}\label{acknowledgements}

An Acknowledgements section is started with \texttt{\textbackslash{}ack}
or \texttt{\textbackslash{}acks} for \emph{Acknowledgement} or
\emph{Acknowledgements}, respectively. It must be placed just before the
References. Define the content of the acknowledgment section in the YAML
header.

\subsection{Bibliography}\label{bibliography}

Link a \texttt{.bib} document via the YAML header, and bibliography will
be printed at the very end (as usual). Remember to include a
\texttt{\#Bibliography} section.

The default bibliography style is provided by Wiley as in
\texttt{wileyj.bst}. It is possible to provide a custom style by
providing a \texttt{.csl}/\texttt{.bst} in the
\texttt{bibliographystyle} field of the YAML header.

\subsection{Double Spacing}\label{double-spacing}

If you need to double space your document for submission please use the
\texttt{doublespace} option in the header.

\section{Copyright Statement}\label{copyright-statement}

Please be aware that the use of this \LaTeX class file is governed by
the following conditions.

\subsection{Copyright}\label{copyright}

Copyright 2010 John Wiley \& Sons, Ltd, The Atrium, Southern Gate,
Chichester, West Sussex, PO19 8SQ, UK. All rights reserved.

\subsection{Rules of Use}\label{rules-of-use}

This class file is made available for use by authors who wish to prepare
an article for publication in \emph{Statistics in Medicine} published by
John Wiley \& Sons, Ltd. The user may not exploit any part of the class
file commercially.

This class file is provided on an \emph{as is} basis, without warranties
of any kind, either express or implied, including but not limited to
warranties of title, or implied warranties of merchantablility or
fitness for a particular purpose. There will be no duty on the
author{[}s{]} of the software or John Wiley \& Sons, Ltd to correct any
errors or defects in the software. Any statutory rights you may have
remain unaffected by your acceptance of these rules of use.

\acks We thank \href{https://stackoverflow.com/}{Stack Overflow} for the
invaluable support to our research.

\bibliographystyle{wileyj}
\bibliography{bibfile.bib}

\end{document}
